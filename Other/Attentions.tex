\subsection{浮点}
  \begin{enumerate}
    \item 浮点初始化memset(d,0x7f,sizeof(d));
    \item 浮点数比大小
      \begin{itemize}
          \item 相等 if ( fabs (a-b) <= eps )
          \item 大于 if ( a>b \&\& fabs (a-b) > eps )
          \item 小于 if ( a<b \&\& fabs (a-b) > eps )
      \end{itemize}
  \end{enumerate}

\subsection{整数类型范围}
  \begin{enumerate}
    \item 255:1111 1111B
    \item 65535:2\^{}16-1, 16bit无符号整数
    \item 2147483647:2\^{}31-1, 32bit带符号整数的最大值
    \item 4294967296:2\^{}32, 32bit无符号整数的最大值
    \item 92233720368547758072:2\^{}63-1, 64bit带符号整数的最大值
    \item 1061109567:0x3f3f3f3f, int inf, 略大于1e9
    \item 4557430888798830399:0x3f3f3f3f3f3f3f3f, ll inf      
  \end{enumerate}

\subsection{热身赛}
  \begin{enumerate}
    \item 测pbds
    \item python3计算器
  \end{enumerate}

\subsection{计算器}
  \begin{enumerate}
    \item 终端
      \begin{itemize}
        \item 分解素因数factor num
        \item 逆串rev+enter string
      \end{itemize}
    \item python3
      \begin{itemize}
        \item from fractions import * [Fraction,gcd]
          \begin{itemize}
            \item 最简分数fraction(a,b)
            \item gcd(a,b)
          \end{itemize}
        \item from math import *
          \begin{itemize}
            \item 阶乘factorial(num)
          \end{itemize}
      \end{itemize}
  \end{enumerate}

\subsection{Attention}
  \begin{enumerate}
    \item 审题
      \begin{itemize}
        \item \textbf{读新题的优先级高于一切}
        \item \textbf{注意限制条件},不清楚的善用Clarification
        \item 读完题、交题前都要看一遍clarification
        \item 每题至少两人确认题意
      \end{itemize}
    \item 做题
      \begin{enumerate}
        \item \textbf{开题}
          \begin{itemize}
            \item 构造不要开场做
            \item 想不出优雅复杂度但过了很多队的\textbf{暴力}莽一莽,单车变摩托

          \end{itemize}
        \item \textbf{上机}
          \begin{itemize}
            \item 和队友确认做法
            \item 有猜想性质的后面写
            \item 写了半小时以上的考虑是否弃题
            \item \textbf{细节和公式纸上写好},不要越码越乱
            \item 中后期题考虑一人写一人辅助,及时发现手误
            \item 多题要写时,容易码、码量小、想得无敌清楚的优先
          \end{itemize}
        \item \textbf{交题}
          \begin{itemize}
            \item 检查初始化和清空
            \item 取模的输出前再模一次
            \item claris: 检查solve(n,m)==solve(m,n)?
            \item spj的题目提交前也应尽量与样例完全一致
            \item claris: 舍入输出若abs不超过eps,需要强行设置0防止-0.000000的出现
          \end{itemize}
      \end{enumerate}
    \item 打印
      \begin{itemize}
        \item 交完题目马上打印并让机
        \item 打表时想清楚打哪些量,代码乱改前注意备份。善用打印,保留代码。
      \end{itemize}
    \item \textbf{心态}:签到莫急,最后半小时不要慌。
  \end{enumerate}

\subsection{Debug}
  \begin{enumerate}
    \item 初始化,清空图,0和n等边界,mem里面sizeof(int)还是ll
    \item for里是给本层循环变量++咩?
    \item 区间l,r为防坑:if(l>r) swap(l,r);
    \item 考虑小数据有没有发生突变的地方
    \item 注意板子有没有哪里要改ll
    \item inf的大小符不符合
  \end{enumerate}

\subsection{打表找规律}
  \begin{enumerate}
    \item 直接找规律
    \item 差分后找规律
    \item 找积性
    \item 点阵打表
    \item 相除
    \item 循环节
    \item 凑量纲
    \item 猜想满足P(n)f(n)=Q(n)f(n-2)+R(n)f(n-1)+C, 其中P、Q、R为关于n二次多项式
  \end{enumerate}

\subsection{优化}
  \begin{enumerate}
    \item 数论
      \begin{itemize}
        \item 分块加速O(sqrt(n))
        \item 枚举除数、调和级数O(log(n))
        \item floor函数求和、ceil函数求和(hdu6134)
        \item getpre里的取模,以及连续取模注意顺序,还有爆精度取模和式子i从2开始(n/i)
        \item \textbf{WA太久或出不了,考虑公式是否错误}
      \end{itemize}
    \item cdq分治
    \item 树上点分治
    \item 一般分块
  \end{enumerate}
