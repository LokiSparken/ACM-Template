% \twocolumn  															%是否分栏
%\documentclass{article}													%中文可能导致乱码
\documentclass[twoside,sub3section,UTF8]{ctexart}						%可使用中文
%====================导言区==========================
%封面设置
\title{Simple template}									%标题
\author{Sparken}										%作者
\CTEXoptions[today=old]								%使显示英文日期
\date{\today}											%显示日期

%页边距设置
\usepackage{geometry}
\geometry{left=2.5cm,right=2.5cm,top=2.5cm,bottom=2.5cm}

%页眉页脚设置
\usepackage{fancyhdr}
\pagestyle{fancy}
\lhead{Queuing}
\chead{}
\rhead{}
\lfoot{}
\cfoot{\thepage}
\rfoot{}
\renewcommand{\headrulewidth}{0.4pt}
\renewcommand{\headwidth}{\textwidth}
\renewcommand{\footrulewidth}{0pt}

%标题居左
\CTEXsetup[format={\Large\bfseries}]{section}

%模板文件导入
\usepackage{minted}
\setminted{tabsize=4,breaklines}
%\setminted{tabsize=4,breaklines,linenos}			%选项设置:tabsize, 语法高亮, 行号
%格式:\inputminted{c++}{"~/*.cpp}
%{minted} manual: https://github.com/gpoore/minted/blob/master/source/minted.pdf

%数学公式相关
\usepackage{amsmath,epsfig,amssymb,subfigure,bm,dsfont}

%目录超链接
\usepackage[colorlinks,linkcolor=black]{hyperref}

%====================以上导言区==========================
\begin{document}
\maketitle

\newpage
\tableofcontents 														%创建目录
	
\newpage 																%换页 
\section{默背一万遍的注意事项}
	\subsection{浮点}
  \begin{enumerate}
    \item 浮点初始化memset(d,0x7f,sizeof(d));
    \item 浮点数比大小
      \begin{itemize}
          \item 相等 if ( fabs (a-b) <= eps )
          \item 大于 if ( a>b \&\& fabs (a-b) > eps )
          \item 小于 if ( a<b \&\& fabs (a-b) > eps )
      \end{itemize}
  \end{enumerate}

\subsection{整数类型范围}
  \begin{enumerate}
    \item 255:1111 1111B
    \item 65535:2\^{}16-1, 16bit无符号整数
    \item 2147483647:2\^{}31-1, 32bit带符号整数的最大值
    \item 4294967296:2\^{}32, 32bit无符号整数的最大值
    \item 92233720368547758072:2\^{}63-1, 64bit带符号整数的最大值
    \item 1061109567:0x3f3f3f3f, int inf, 略大于1e9
    \item 4557430888798830399:0x3f3f3f3f3f3f3f3f, ll inf      
  \end{enumerate}

\subsection{热身赛}
  \begin{enumerate}
    \item 测pbds
    \item python3计算器
  \end{enumerate}

\subsection{计算器}
  \begin{enumerate}
    \item 终端
      \begin{itemize}
        \item 分解素因数factor num
        \item 逆串rev+enter string
      \end{itemize}
    \item python3
      \begin{itemize}
        \item from fractions import * [Fraction,gcd]
          \begin{itemize}
            \item 最简分数fraction(a,b)
            \item gcd(a,b)
          \end{itemize}
        \item from math import *
          \begin{itemize}
            \item 阶乘factorial(num)
          \end{itemize}
      \end{itemize}
  \end{enumerate}

\subsection{Attention}
  \begin{enumerate}
    \item 审题
      \begin{itemize}
        \item \textbf{读新题的优先级高于一切}
        \item \textbf{注意限制条件},不清楚的善用Clarification
        \item 读完题、交题前都要看一遍clarification
        \item 每题至少两人确认题意
      \end{itemize}
    \item 做题
      \begin{enumerate}
        \item \textbf{开题}
          \begin{itemize}
            \item 构造不要开场做
            \item 想不出优雅复杂度但过了很多队的\textbf{暴力}莽一莽,单车变摩托

          \end{itemize}
        \item \textbf{上机}
          \begin{itemize}
            \item 和队友确认做法
            \item 有猜想性质的后面写
            \item 写了半小时以上的考虑是否弃题
            \item \textbf{细节和公式纸上写好},不要越码越乱
            \item 中后期题考虑一人写一人辅助,及时发现手误
            \item 多题要写时,容易码、码量小、想得无敌清楚的优先
          \end{itemize}
        \item \textbf{交题}
          \begin{itemize}
            \item 检查初始化和清空
            \item 取模的输出前再模一次
            \item claris: 检查solve(n,m)==solve(m,n)?
            \item spj的题目提交前也应尽量与样例完全一致
            \item claris: 舍入输出若abs不超过eps,需要强行设置0防止-0.000000的出现
          \end{itemize}
      \end{enumerate}
    \item 打印
      \begin{itemize}
        \item 交完题目马上打印并让机
        \item 打表时想清楚打哪些量,代码乱改前注意备份。善用打印,保留代码。
      \end{itemize}
    \item \textbf{心态}:签到莫急,最后半小时不要慌。
  \end{enumerate}

\subsection{Debug}
  \begin{enumerate}
    \item 初始化,清空图,0和n等边界,mem里面sizeof(int)还是ll
    \item for里是给本层循环变量++咩?
    \item 区间l,r为防坑:if(l>r) swap(l,r);
    \item 考虑小数据有没有发生突变的地方
    \item 注意板子有没有哪里要改ll
    \item inf的大小符不符合
  \end{enumerate}

\subsection{打表找规律}
  \begin{enumerate}
    \item 直接找规律
    \item 差分后找规律
    \item 找积性
    \item 点阵打表
    \item 相除
    \item 循环节
    \item 凑量纲
    \item 猜想满足P(n)f(n)=Q(n)f(n-2)+R(n)f(n-1)+C, 其中P、Q、R为关于n二次多项式
  \end{enumerate}

\subsection{优化}
  \begin{enumerate}
    \item 数论
      \begin{itemize}
        \item 分块加速O(sqrt(n))
        \item 枚举除数、调和级数O(log(n))
        \item floor函数求和、ceil函数求和(hdu6134)
        \item getpre里的取模,以及连续取模注意顺序,还有爆精度取模和式子i从2开始(n/i)
        \item \textbf{WA太久或出不了,考虑公式是否错误}
      \end{itemize}
    \item cdq分治
    \item 树上点分治
    \item 一般分块
  \end{enumerate}

	% \inputminted{markdown}{"Other/headers.cpp"}
% 	\newpage
% 	\subsection{配置Vim setting}
% 		\noindent==========================================\\
% 		14行基本设置\\
% 		syntax on\\
% 		set cindent\\
% 		set nu		\\
% 		set shortmess=atI	\\
% 		set tabstop=4\\
% 		set shiftwidth=4\\
% 		set confirm\\
% 		set mouse=a\\
% 
% 		map<C-A> ggVG"+y\\
% 		map <F5> :call Run()<CR>\\
% 		func! Run()\par
% 			exec "w"\par
% 			exec "!g++ -Wall \% -o \%<"\par
% 			exec "!time ./\%<"\\
% 		endfunc\\
% 		==========================================\\
% 		括号补全\\
% 		inoremap \{ \{<CR>\}<ESC>kA<CR>\\
% 		inoremap ( ()<ESC>i\\
% 		inoremap [ []<ESC>i\\
% 		inoremap " ""<ESC>i\\
% 		跳转行末\\
% 		inoremap <C- \ > <End>\\
% 		==========================================\\
% 		【快捷键】\\
% 		:nu行跳转
% 		/text查找text,n查找下一个,N查找前一个\\
% 		u撤销,U撤销对整行的操作\\
% 		Ctrl+r撤销的撤销\\

\newpage
\section{图论Graph Theory}
	\subsection{最短路The shortest path}
		\subsubsection{Dijkstra}
			\inputminted{c++}{"Gragh Theory/The shortest path/dijkstra.cpp"}
		\subsubsection{Spfa}
			\inputminted{c++}{"Gragh Theory/The shortest path/spfa.cpp"}
		% \subsubsection{Floyd}
		% \inputminted{c++}{"Gragh Theory/The shortest path/floyd.cpp"}
		\subsubsection{次短路}
			\inputminted{c++}{"Gragh Theory/The shortest path/secdij.cpp"}
		\subsubsection{第K短路}
			\inputminted{c++}{"Gragh Theory/The shortest path/Astar.cpp"}

	% \subsection{生成树Spanning tree}
		\subsection{最小生成树Minimum spanning tree}
			\subsubsection{Kruskal}
				\inputminted{c++}{"Gragh Theory/MST/kruskal.cpp"}
			\subsubsection{Prim}
				\inputminted{c++}{"Gragh Theory/MST/prim.cpp"}
		\subsection{次小生成树}
			\inputminted{c++}{"Gragh Theory/MST/secmst.cpp"}
		% \subsubsection{最小树形图Minimum Arborescence}


	\subsection{网络流Network flow}
		\subsubsection{最大流-Dinic}
			\inputminted{c++}{"Gragh Theory/Network Flow/dinic.cpp"}
		\subsubsection{最大流-ISAP}
			\inputminted{c++}{"Gragh Theory/Network Flow/ISAP.cpp"}
		\subsubsection{最小费用最大流-EdmondsKarp}
			\inputminted{c++}{"Gragh Theory/Network Flow/MCMF.cpp"}
		\subsubsection{建图-有上下界的可行流}
			\inputminted{c++}{"Gragh Theory/Network Flow/lowup.cpp"}

	% \subsection{二分图}
	% 	\subsubsection{概念公式}
	% 		\inputminted{c++}{"Gragh Theory/Bipartite gragh/emmm.cpp"}
	% 	\subsubsection{最大匹配-匈牙利}
	% 		\inputminted{c++}{"Gragh Theory/Bipartite gragh/hungary.cpp"}
	% 		\inputminted{c++}{"Gragh Theory/Bipartite gragh/hungarymatrix.cpp"}
		%\subsubsection{最大匹配-Hopcroft-Karp}
		%\inputminted{c++}{"Gragh Theory/Bipartite gragh/HK.cpp"}
		%\subsubsection{最大权完美匹配-KM}
		%\inputminted{c++}{"Gragh Theory/Bipartite gragh/KM.cpp"}
	
	\subsection{强连通缩点tarjan}
		\inputminted{c++}{"Gragh Theory/SCC.cpp"}

	\subsection{最近公共祖先LCA}
		\subsubsection{tarjan}
			\inputminted{c++}{"Gragh Theory/LeastCommonAncestors/tarjan.cpp"}
		\subsubsection{ST表}
			\inputminted{c++}{"Gragh Theory/LeastCommonAncestors/dfs+ST.cpp"}

	\subsection{欧拉回路}
		\subsubsection{判定}
			% \inputminted{c++}{"Gragh Theory/EulerRoad/judge.md"}
		\subsubsection{求解}
			\inputminted{c++}{"Gragh Theory/EulerRoad/construction.cpp"}

	% \subsection{哈密顿回路}


\newpage
\section{数据结构Data Structure}
	\subsection{并查集Union-Find Set}
		\inputminted{c++}{"Data Structure/union-find-set.cpp"}
	\subsection{拓扑排序Topological Sorting}
		\inputminted{c++}{"Data Structure/topo.cpp"}
	\subsection{树状数组}
		\inputminted{c++}{"Data Structure/BinaryIndexedTree.cpp"}
	\subsection{RMQ}
		\inputminted{c++}{"Data Structure/RMQ.cpp"}
	\subsection{表达式树}
		\inputminted{c++}{"Data Structure/ExpressionTree.cpp"}
	\subsection{线段树Segment Tree}
		\subsubsection{基础}
			\inputminted{c++}{"Data Structure/segmentTree/segmentTree.cpp"}
		\subsubsection{维护线性变化}
			\inputminted{c++}{"Data Structure/segmentTree/pushdown.cpp"}
	\subsection{可持久化数据结构}
		\subsubsection{01字典树}
			\inputminted{c++}{"Data Structure/Presistence/01Dictionary.cpp"}
		\subsubsection{权值线段树}
			\inputminted{c++}{"Data Structure/Presistence/HJT.cpp"}
	\subsection{树链剖分HeavyLightDecomposition}
		\inputminted{c++}{"Data Structure/HeavyLightDecomposition.cpp"}
	\subsection{伸展树splay}
		\subsubsection{维护序列}
			\inputminted{c++}{"Data Structure/splay.cpp"}
		\subsubsection{平衡树(cc)}
			\inputminted{c++}{"Data Structure/Rank tree.cpp"}
	\subsection{Treap}
		\inputminted{c++}{"Data Structure/treap.cpp"}
		
\newpage
\section{数学Math}
	% %\begin{eqnarray}
%I_{a_1a_2 \cdots a_n}&=&
%\left\{
%\begin{array}{lll}
%1, \ \ a_1=a_2=\cdots=a_n. \\
%0, \ \ otherwise.
%\end{array}
%\right.
%\end{eqnarray}
\subsection{定理公式与结论Conclusions}
1、费马小定理:$ a^{p-1} \equiv \ 1 \ (mod \ p) $ \quad 当 $(a,p)=1$ \par
2、欧拉降幂公式:
    % a^{x} \left ( \  mod \ p \right \ ) = \left\{\begin{matrix}
    % a^{x \ \% \ \varphi (p) } & (a,p)=1. & \\ 
    % a^{x \ \% \ \varphi (p) \ + \ \varphi (p)} & (a,p)\not=1, \ x \ge \varphi(p). & \\ 
    % a^x & (a,p)\not=1, \ x< \varphi(p). & 
    % \end{matrix}\right.
        \begin{eqnarray}
            a^x \ (mod \ p) \ = &
            \left\{
            \begin{array}{lr}
                a^{x \% \varphi(p)}, \ \ (a,p)=1. \\
                a^{x \% \varphi(p) + \varphi(p)}, \ \ (a,p)\not=1, \ x \ge \varphi(p). \\
                a^x, \ \ (a,p)\not=1, \ x< \varphi(p).
            \end{array}
            \right.
        \end{eqnarray}
    % $$ a^x \ (mod \ p) \ =  $$
    %     \begin{equation}
    %     \left\{
    %                  \begin{array}{lr}
    %                  x=a^{x \% \varphi(p)}, & (a,p)=1 \\
    %                  y=a^{x \% \varphi(p) + \varphi(p)}, & (a,p)\not=1, \ x \ge \varphi(p)\\
    %                  z=a^x, & (a,p)\not=1, \ x< \varphi(p)
    %                  \end{array}
    %     \right.
    %     \end{equation}
	\subsection{快速乘-快速幂}
		\inputminted{c++}{"Maths/fastmul.cpp"}
		\inputminted{c++}{"Maths/fastpow.cpp"}
	\subsection{矩阵快速幂}
		\inputminted{c++}{"Maths/MatrixFastpow.cpp"}	
	\subsection{扩展欧几里得}
		\inputminted{c++}{"Maths/exgcd.cpp"}
	\subsection{筛法求素数}
		\subsubsection{埃式筛}
			\inputminted{c++}{"Maths/Prime/EratosthenesSieve.cpp"}
		\subsection{欧拉筛线性筛}
			\inputminted{c++}{"Maths/Prime/EularSieve.cpp"}
		\subsubsection{区间筛}
			\inputminted{c++}{"Maths/Prime/SegmentSieve.cpp"}
	\subsection{逆元}
		\subsubsection{模n下a的逆元}
			\inputminted{c++}{"Maths/Rev/modnreva.cpp"}
		\subsubsection{线性求逆元}
			\inputminted{c++}{"Maths/Rev/rev.cpp"}
	\subsection{欧拉函数}
		\inputminted{c++}{"Maths/Elur.cpp"}
	\subsection{中国剩余定理求同余方程组}
		\subsubsection{素数}
			\inputminted{c++}{"Maths/CRT(prime).cpp"}
		\subsubsection{非素数}
			\inputminted{c++}{"Maths/CRT(notprime).cpp"}
	\subsection{数值计算}
		\subsubsection{FFT}
			\inputminted{c++}{"Maths/FFT/FFT.cpp"}
		\subsubsection{FFT二进制反转问题}
			\inputminted{c++}{"Maths/FFT/FFTbitrev.cpp"}
		\subsubsection{NTT}
			\inputminted{c++}{"Maths/FFT/NTT.cpp"}
	\subsection{卢卡斯Lucas}
		\subsubsection{Lucas}
			\inputminted{c++}{"Maths/Lucas/lucas.cpp"}
		\subsubsection{扩展卢卡斯ExLucas} 
			\inputminted{c++}{"Maths/Lucas/exlucas.cpp"}
	\subsection{线性基}
		\inputminted{c++}{"Maths/xianXingJi.cpp"}
	\subsection{自适应辛普森}
		\inputminted{c++}{"Maths/Simpson.cpp"}
	\subsection{高斯消元GauseElimination}
		\inputminted{c++}{"Maths/GaussElimination.cpp"}
	\subsection{对角阵GaussJordan}
		\inputminted{c++}{"Maths/GaussJordan.cpp"}
	\subsection{米勒罗宾素数测试MillerRabin}
		\inputminted{c++}{"Maths/MillerRabin.cpp"}
	\subsection{模方程(可非素数)}
		\inputminted{c++}{"Maths/modEquation.cpp"}

	%\subsection{线性递推BM}
	%\subsection{结论}
	%1、两素数最大不能生成数为m*n-m-n
	%2、

\section{字符串String}
	\subsection{字典树Trie}
		\inputminted{c++}{"String/Trie.cpp"}
	\subsection{KMP}
		\inputminted{c++}{"String/KMP.cpp"}
	% \subsection{扩展KMP}
	% \inputminted{c++}{"String/EXKMP.cpp"}
	% \subsection{最长回文子串Manacher}
	% \inputminted{c++}{"String/Manacher.cpp"}
	\subsection{AC自动机}
		\inputminted{c++}{"String/AC.cpp"}
	\subsection{回文树}
		\inputminted{c++}{"String/PAM.cpp"}
	
\section{动态规划Dynamic Programme}
	\subsection{背包}
		\inputminted{c++}{"Dynamic Programme/bag.cpp"}
	\subsection{旅行商TSP}
		\inputminted{c++}{"Dynamic Programme/TSP.cpp"}
	\subsection{数位dp}
		\inputminted{c++}{"Dynamic Programme/digit.cpp"}

\section{计算几何Computation Geometry}
	%\subsection{基础函数}

	%\subsection{凸包}

	\subsection{圆Circle}
		\subsubsection{求两圆交点}
		\inputminted{c++}{"Computation Geometry/Circle/twopoint.cpp"}

\newpage
\section{其它Other}
	\subsection{莫队}
	\inputminted{c++}{"Other/Mos.cpp"}
	\subsection{离散化}
	\inputminted{c++}{"Other/discretization.cpp"}
	% \subsection{简单大数}
	% \inputminted{c++}{"IO/BigInt.cpp"}	
	\subsection{STL}
	\inputminted{c++}{"Other/STL.cpp"}
	\subsection{pbds}
	\inputminted{c++}{"Other/pbds.cpp"}
	%\subsection{数组模拟乘法}
	%\inputminted{c++}{"Other/jiechengmoni.cpp"}

\newpage
\section{输入输出IO}
	% \subsection{简陋IO挂}
	% 	\inputminted{c++}{"IO/IO.cpp"}
	\subsection{Python输入输出}
		\inputminted{python}{"IO/IO.py"}
	\subsection{Java高精度BigDecimal}
		\inputminted{java}{"IO/IO.java"}

\end{document}