% \twocolumn  															%是否分栏
%\documentclass{article}													%中文可能导致乱码
\documentclass[twoside,sub3section,UTF8]{ctexart}						%可使用中文
%====================导言区==========================
%封面设置
\title{Simple template}									%标题
\author{Sparken}										%作者
\CTEXoptions[today=old]								%使显示英文日期
\date{\today}											%显示日期

%页边距设置
\usepackage{geometry}
\geometry{left=2.5cm,right=2.5cm,top=2.5cm,bottom=2.5cm}

%页眉页脚设置
\usepackage{fancyhdr}
\pagestyle{fancy}
\lhead{Mirai}
\chead{}
\rhead{}
\lfoot{}
\cfoot{\thepage}
\rfoot{}
\renewcommand{\headrulewidth}{0.4pt}
\renewcommand{\headwidth}{\textwidth}
\renewcommand{\footrulewidth}{0pt}
 
%标题居左
\CTEXsetup[format={\Large\bfseries}]{section}

%模板文件导入
\usepackage{minted}
\setminted{tabsize=4,breaklines}
%\setminted{tabsize=4,breaklines,linenos}			%选项设置:tabsize, 语法高亮, 行号
%格式:\inputminted{c++}{"~/*.cpp}
%{minted} manual: https://github.com/gpoore/minted/blob/master/source/minted.pdf

%数学公式相关
\usepackage{amsmath,epsfig,amssymb,subfigure,bm,dsfont}

%目录超链接
\usepackage[colorlinks,linkcolor=black]{hyperref}
	
%====================以上导言区==========================
\begin{document}
\maketitle

\newpage
\tableofcontents 														%创建目录

\newpage 																%换页
\section{Pretreatment}												%一级标题
	\subsection{头文件Headers and constants}
	\inputminted[breaklines]{c++}{"Other/headers.cpp"}
	\newpage
	\subsection{配置Vim setting}
		\noindent==========================================\\
		14行基本设置\\
		syntax on\\
		set cindent\\
		set nu		\\
		set shortmess=atI	\\
		set tabstop=4\\
		set shiftwidth=4\\
		set confirm\\
		set mouse=a\\
\\
		map<C-A> ggVG"+y\\
		map <F5> :call Run()<CR>\\
		func! Run()\par
			exec "w"\par
			exec "!g++ -Wall \% -o \%<"\par
			exec "!time ./\%<"\\
		endfunc\\
		==========================================\\
		括号补全\\
		inoremap \{ \{<CR>\}<ESC>kA<CR>\\
		inoremap ( ()<ESC>i\\
		inoremap [ []<ESC>i\\
		inoremap " ""<ESC>i\\
		跳转行末\\
		inoremap <C- \ > <End>\\
		==========================================\\
		【快捷键】\\
		:nu行跳转
		/text查找text,n查找下一个,N查找前一个\\
		u撤销,U撤销对整行的操作\\
		Ctrl+r撤销的撤销\\

\newpage
\section{图论Graph Theory}
	\subsection{最短路The shortest path}
		\subsubsection{Dijkstra}
		\inputminted[breaklines]{c++}{"Gragh Theory/The shortest path/dijkstra.cpp"}
		\subsubsection{Spfa}
		\inputminted[breaklines]{c++}{"Gragh Theory/The shortest path/spfa.cpp"}
		% \subsubsection{Floyd}
		% \inputminted[breaklines]{c++}{"Gragh Theory/The shortest path/floyd.cpp"}
		\subsubsection{次短路}
		\inputminted[breaklines]{c++}{"Gragh Theory/The shortest path/secdij.cpp"}
		\subsubsection{第K短路}
		\inputminted[breaklines]{c++}{"Gragh Theory/The shortest path/Astar.cpp"}

	\subsection{生成树Spanning tree}
		\subsubsection{最小生成树Minimum spanning tree}
		\inputminted[breaklines]{c++}{"Gragh Theory/MST/kruskal.cpp"}
		\inputminted[breaklines]{c++}{"Gragh Theory/MST/prim.cpp"}
		\subsubsection{次小生成树}
		\inputminted{c++}{"Gragh Theory/MST/secmst.cpp"}
		% \subsubsection{最小树形图Minimum Arborescence}
		

	\subsection{网络流Network flow}
		\subsubsection{最大流-Dinic}
		\inputminted{c++}{"Gragh Theory/Network Flow/dinic.cpp"}
		\subsubsection{最大流-ISAP}
		\inputminted{c++}{"Gragh Theory/Network Flow/ISAP.cpp"}
		\subsubsection{最小费用最大流-EdmondsKarp}
		\inputminted{c++}{"Gragh Theory/Network Flow/MCMF.cpp"}
		\subsubsection{建图-有上下界的可行流}
		\inputminted{c++}{"Gragh Theory/Network Flow/lowup.cpp"}

	\subsection{二分图}
		\subsubsection{概念公式}
		\inputminted{c++}{"Gragh Theory/Bipartite gragh/emmm.cpp"}
		\subsubsection{最大匹配-匈牙利}
			\inputminted{c++}{"Gragh Theory/Bipartite gragh/hungary.cpp"}
			\inputminted{c++}{"Gragh Theory/Bipartite gragh/hungarymatrix.cpp"}
		%\subsubsection{最大匹配-Hopcroft-Karp}
		%\inputminted{c++}{"Gragh Theory/Bipartite gragh/HK.cpp"}
		%\subsubsection{最大权完美匹配-KM}
		%\inputminted{c++}{"Gragh Theory/Bipartite gragh/KM.cpp"}

	\subsection{强连通缩点tarjan}
	\inputminted{c++}{"Gragh Theory/SCC.cpp"}

	\subsection{最近公共祖先LCA}
		\subsubsection{tarjan}
		\inputminted{c++}{"Gragh Theory/LeastCommonAncestors/tarjan.cpp"}
		\subsubsection{dfs+ST}
		\inputminted{c++}{"Gragh Theory/LeastCommonAncestors/dfs+ST.cpp"}

	\subsection{欧拉回路}
		\subsubsection{判定}
		%\inputminted{c++}{"Gragh Theory/ouraroad.cpp"}
		\subsubsection{求解}
		%\inputminted{c++}{"Gragh Theory/ouraroadsolve.cpp"}

	\subsection{哈密顿回路}


\newpage
\section{数据结构Data Structure}
	\subsection{并查集Union-Find Set}
	\inputminted{c++}{"Data Structure/union-find-set.cpp"}
	\subsection{拓扑排序Topological Sorting}
	\inputminted{c++}{"Data Structure/topo.cpp"}
	\subsection{树状数组}
		\inputminted{c++}{"Data Structure/BinaryIndexedTree.cpp"}
	\subsection{RMQ}
		\inputminted{c++}{"Data Structure/RMQ.cpp"}
	\subsection{线段树Segment Tree}
		\inputminted{c++}{"Data Structure/segmentTree.cpp"}
	\subsection{树链剖分HeavyLightDecomposition}
		\inputminted{c++}{"Data Structure/HeavyLightDecomposition.cpp"}
	\subsection{伸展树splay}
		\subsubsection{维护序列}
		\inputminted{c++}{"Data Structure/splay.cpp"}
		\subsubsection{平衡树}


\newpage
\section{数学Math}
	%\begin{eqnarray}
%I_{a_1a_2 \cdots a_n}&=&
%\left\{
%\begin{array}{lll}
%1, \ \ a_1=a_2=\cdots=a_n. \\
%0, \ \ otherwise.
%\end{array}
%\right.
%\end{eqnarray}
\subsection{定理公式与结论Conclusions}
1、费马小定理:$ a^{p-1} \equiv \ 1 \ (mod \ p) $ \quad 当 $(a,p)=1$ \par
2、欧拉降幂公式:
    % a^{x} \left ( \  mod \ p \right \ ) = \left\{\begin{matrix}
    % a^{x \ \% \ \varphi (p) } & (a,p)=1. & \\ 
    % a^{x \ \% \ \varphi (p) \ + \ \varphi (p)} & (a,p)\not=1, \ x \ge \varphi(p). & \\ 
    % a^x & (a,p)\not=1, \ x< \varphi(p). & 
    % \end{matrix}\right.
        \begin{eqnarray}
            a^x \ (mod \ p) \ = &
            \left\{
            \begin{array}{lr}
                a^{x \% \varphi(p)}, \ \ (a,p)=1. \\
                a^{x \% \varphi(p) + \varphi(p)}, \ \ (a,p)\not=1, \ x \ge \varphi(p). \\
                a^x, \ \ (a,p)\not=1, \ x< \varphi(p).
            \end{array}
            \right.
        \end{eqnarray}
    % $$ a^x \ (mod \ p) \ =  $$
    %     \begin{equation}
    %     \left\{
    %                  \begin{array}{lr}
    %                  x=a^{x \% \varphi(p)}, & (a,p)=1 \\
    %                  y=a^{x \% \varphi(p) + \varphi(p)}, & (a,p)\not=1, \ x \ge \varphi(p)\\
    %                  z=a^x, & (a,p)\not=1, \ x< \varphi(p)
    %                  \end{array}
    %     \right.
    %     \end{equation}
	\subsection{快速乘-快速幂}
	\inputminted{c++}{"Maths/fastmul.cpp"}
	\inputminted{c++}{"Maths/fastpow.cpp"}
	\subsection{矩阵快速幂}
	\inputminted{c++}{"Maths/MatrixFastpow.cpp"}
	\subsection{扩展欧几里得}
	\inputminted{c++}{"Maths/exgcd.cpp"}
	\subsection{欧拉函数}
	\inputminted{c++}{"Maths/Elur.cpp"}
	\subsection{中国剩余定理求同余方程组}
		\subsubsection{素数}
		\inputminted{c++}{"Maths/CRT(prime).cpp"}
		\subsubsection{非素数}
		\inputminted{c++}{"Maths/CRT(notprime).cpp"}
	%\subsection{FFT}
	%\subsection{卢卡斯Lucas}
	%\subsection{线性递推BM}
	%\subsection{结论}
	%1、两素数最大不能生成数为m*n-m-n
	%2、

\section{字符串String}
	\subsection{字典树Trie}
	\inputminted{c++}{"String/Trie.cpp"}
	\subsection{KMP}
	\inputminted{c++}{"String/KMP.cpp"}
	% \subsection{扩展KMP}
	% \inputminted{c++}{"String/EXKMP.cpp"}
	% \subsection{最长回文子串Manacher}
	% \inputminted{c++}{"String/Manacher.cpp"}
	\subsection{AC自动机}
	\inputminted{c++}{"String/AC.cpp"}
	
\section{动态规划}
	\subsection{背包}
	\inputminted{c++}{"Dynamic Programme/bag.cpp"}
	\subsection{旅行商TSP}
	\inputminted{c++}{"Dynamic Programme/TSP.cpp"}
	% \subsection{数位dp}
	% \inputminted{c++}{"Dynamic Programme/digit.cpp"}

%\section{计算几何}
	%\subsection{基础函数}

	%\subsection{凸包}


\newpage
\section{其它Other}
	\subsection{莫队}
	\inputminted{c++}{"Other/Mos.cpp"}
	\subsection{离散化}
	\inputminted{c++}{"Other/discretization.cpp"}
	\subsection{简单大数}
	\inputminted{c++}{"IO/BigInt.cpp"}	
	\subsection{STL}
	\inputminted{c++}{"Other/STL.cpp"}
	\subsection{pbds}
	\inputminted{c++}{"Other/pbds.cpp"}
	%\subsection{数组模拟乘法}
	%\inputminted{c++}{"Other/jiechengmoni.cpp"}

\newpage
\section{输入输出IO}
	\subsection{简陋IO挂}
	\inputminted{c++}{"IO/IO.cpp"}
	\subsection{Python输入输出}
	\inputminted{python}{"IO/IO.py"}
	\subsection{Java高精度BigDecimal}
	\inputminted{java}{"IO/IO.java"}

\end{document}
